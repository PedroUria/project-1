\documentclass[man]{apa6}

\usepackage{lipsum}

\usepackage[american]{babel}

\usepackage{csquotes}
\usepackage[style=apa,sortcites=true,sorting=nyt,backend=biber]{biblatex}
\DeclareLanguageMapping{american}{american-apa}
\addbibresource{project1.bib}

\title{Analysis of Customer Churn at Telco}

\shorttitle{Customer Churn}

\author{Pedro Uria, Sean Pili, and Zachary Buckley}

\affiliation{George Washington University}

\abstract{
  TODO: This is the abstract.
}

\begin{document}
\maketitle

\section{Introduction and Background Research}
\paragraph{Churn}


\paragraph{Dataset}
The Telco Customer Churn Dataset we'll be using throughout this paper contains information about the customers from an unidentified telecommunications company. The data was uploaded to kaggle.com by a user named blastchar. \cite{blastchar_2018} The dataset originates from a sample dataset provided by IBM for exploring the capabilities of IBM's Watson Analytics services. IBM uses the data in a walkthrough of building a model for predicting whether a customer will leave the company, with the explicit goal of creating better customer retention programs. \cite{ibm_telco_2015} There are a number of sample data sets provided by IBM for similar walkthroughs related to their Watson Analytics system. \cite{ibm_data_2015} We were unable to determine conclusively if the Telco Dataset we're interested in for this paper was based on data from a real telecommunications company, or if the data was synthetically generated for the purposes of demostrating the watson analytics system's capabilities.

\paragraph{Other Research}
B

\section{Research Question, Hypotheses and Methods (variable explanations)}
\paragraph{Research Question}
C
\paragraph{Hypothesis}
D
\paragraph{Methods}
E
\section{Overview of the findings including EDA and model results}
F
\section{A discussion of what conclusions can be drawn as a result of your project.}
G
\newpage
\printbibliography
\end{document}
